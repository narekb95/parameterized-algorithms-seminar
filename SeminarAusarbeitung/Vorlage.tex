% Vorlage f�r Seminar-Ausarbeitungen
%
% Dateiname: Vorlage.tex
% zuletzt ge�ndert: 11. Februar 2016
% Autorin: Nicole Schweikardt

% Definition der Dokument-Klasse, in diesem Falle die Klasse SeminarAusarbeitung,
% die in der Datei SeminarAusarbeitung.cls bereitgestellt wird
\documentclass{SeminarAusarbeitung}

\title{Graph minors\\The Robertson-Seymour theorem}
\author{Narek Bojikian}
\seminar{Parameterized Complexity}
\semester{Winter semester 20/21}
\leitung{Prof.\ Dr.\ Stefan Kratsch} 
\institut{Institut f�r Informatik}
\universitaet{Humboldt-Universit�t zu Berlin}

\usepackage{bbold}
\usepackage{xcolor}
\usepackage{graphicx}
\usepackage{float}
\usepackage{natbib}
\usepackage{caption}
\usepackage{tikz}
\usepackage{bussproofs}
\usepackage{proof}

\newcommand{\conf}{\mathbb{C}}
\newcommand{\negc}[1]{\operatorname{neg}({#1})}
\newcommand{\negd}[1]{\operatorname{neg}_{\mathcal R(d)}({#1})}
\newcommand{\prty}{\operatorname{PARITY}}

\begin{document}

\maketitle
\begin{abstract}
	In this report we summarize the applications of the well-known graph minor theorem to
	parameterized complexity. We represent the notions of nonuniformly fixed-parameter
	tractability and vertex-deletion distance. This work is inspired by section 6.3 from
	Parameterized Algorithms Book by Cygan et al. \cite{DBLP:books/sp/CyganFKLMPPS15}.
\end{abstract}

\section{Introduction}
\subsection{History}
Graph minor theorem was found by Robertson and Seymour in a series of 20 papers to prove the
infamous conjecture of Wagner \cite{} [todo]. Since then this theorem has been crucial in
combinatorics, geometry and graph theory. It has seen applications in different areas of theoretical
computer science. In this report we show applications of this theorem to parameterized complexity
reintroducing the notion of nonuniformly fixed-parameter tractabiliy.

The theorem in short shows that any infinited set of undirected graphs contain two graphs such that
one of them is isomorphic to a minor of the other. This implies that graph properties that are
preserved by taking minors can be described by a finite set of (minimal) forbidden minors. However,
the proof of the theorem is non-constructive and does not deliver a way to find such a list. Such
lists are only known for a few minor-closed classes of graphs such as planar grpahs Kuratowski's
Theorem and outer-planar graphs. For some graph-classes such lists are only partially known and can
be arbitrary large for example, Toroidal graphs (graphs that can be embedded on a torus) have at
least 250,815 known forbidden minors. On the other hand, Robertson and Seymour provide an algorithm,
that given two graphs $G$ and $H$ finds in polynomial time in $G$ and some function in $H$ whether
$H$ is a minor of $G$. Given the complete list of forbidden minors of some class $\mathcal{C}$,
this results yield a polynomial time algorithm to test whether a given graph belong to this class or
not. However, although Robertson and Seymour's theorem prove the existence of such an algorithm, it
is not clear how to derive the algorithm it self, sense there is no clear way to find the list of
forbidden minors of a given graph.

[todo] parameterized complexity

In the rest of this report we give formal definitions of all the mentioned terms and provide formal
statements for the results of Robertson and Seymour. We follow that by their application to
parameterized complexity and two examples from the exercises of the book mentioned above.

\subsection{Preliminaries}
\bibliographystyle{plain}
\bibliography{ref.bib}
\end{document}
