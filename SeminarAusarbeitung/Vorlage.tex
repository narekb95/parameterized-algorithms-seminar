% Vorlage f�r Seminar-Ausarbeitungen
%
% Dateiname: Vorlage.tex
% zuletzt ge�ndert: 11. Februar 2016
% Autorin: Nicole Schweikardt

% Definition der Dokument-Klasse, in diesem Falle die Klasse SeminarAusarbeitung,
% die in der Datei SeminarAusarbeitung.cls bereitgestellt wird
\documentclass{SeminarAusarbeitung}

\title{Graph minors\\The Robertson-Seymour theorem}
\author{Narek Bojikian}
\seminar{Parameterized Complexity}
\semester{Winter semester 20/21}
\leitung{Prof.\ Dr.\ Stefan Kratsch} 
\institut{Institut f�r Informatik}
\universitaet{Humboldt-Universit�t zu Berlin}

\usepackage{bbold}
\usepackage{xcolor}
\usepackage{graphicx}
\usepackage{float}
\usepackage{natbib}
\usepackage{caption}
\usepackage{tikz}
\usepackage{bussproofs}
\usepackage{proof}

\newcommand{\conf}{\mathbb{C}}
\newcommand{\negc}[1]{\operatorname{neg}({#1})}
\newcommand{\negd}[1]{\operatorname{neg}_{\mathcal R(d)}({#1})}
\newcommand{\prty}{\operatorname{PARITY}}

\begin{document}

\maketitle
\begin{abstract}
	In this report we summarize the applications of the well-known graph minor theorem to
	parameterized complexity. We represent the notions of nonuniformly fixed-parameter
	tractability and vertex-deletion distance. This work is inspired by section 6.3 from
	Parameterized Algorithms Book by Cygan et al. \cite{DBLP:books/sp/CyganFKLMPPS15}.
\end{abstract}

\section{Introduction}
\subsection{History}
Graph minor theorem was found by Robertson and Seymour in a series of 20 papers, resulting in a
proof of the well-known conjecture of Wagner. Since then this theorem has been crucial in
combinatorics, geometry and graph theory. It has seen applications in different areas of theoretical
computer science. In this report we show applications of this theorem to parameterized complexity
reintroducing the notion of nonuniformly fixed-parameter tractability.

The theorem in short shows that any infinite set of undirected graphs contain two graphs such that
one of them is isomorphic to a minor of the other. This implies that graph properties that are
preserved by taking minors can be described by a finite set of (minimal) forbidden minors. However,
the proof of the theorem is non-constructive and does not deliver a way to find such a list. Such
lists are only known for a few minor-closed classes of graphs such as planar graphs (Kuratowski's
Theorem) outer-planar graphs and forests. For some graph-classes such lists are only partially known
and can be arbitrary large for example, Toroidal graphs (graphs that can be embedded on a torus)
have at least 250,815 known forbidden minors. On the other hand, Robertson and Seymour provide an
algorithm, that given two graphs $G$ and $H$ finds in polynomial time in $G$ and some function in
$H$ whether $H$ is a minor of $G$. Given the complete list of forbidden minors of some class
$\mathcal{C}$, this results yield a polynomial time algorithm to test whether a given graph belong
to this class or not. However, although Robertson and Seymour's theorem prove the existence of such
an algorithm, it is not clear how to derive the algorithm it self, sense there is no clear way to
find the list of forbidden minors of a given graph.

One application of graph minor theorem, that we will focus on in this report, is in the field of
parameterized complexity. As discussed above, some properties of graphs stay preserved in all minors
of a graph. We have seen that polynomial time algorithms for testing such properties exist. Under
these properties we consider some graph parameters that have their values preserved in all minors
of a given graph. It is not hard to show that a polynomial time algorithm exists, that given a graph
$G$ and an integer value $k$, it tests whether the value of such a parameter in $G$ is at most $k$.
However, since the list of forbidden minors must be hard-coded into the algorithm and we do not know
whether an algorithm exists that generate such a list for each value of the parameter, deciding
whether $G$ has the parameter value $k$ is not quite fixed-parameter tractable. Still, knowing that
such an algorithm exists is a strong notion that a given problem might be fixed-parameter tractable.
We call problems that admit such algorithms nonuniformly fixed-parameter tractable. As we shall see
later in this report, a family of parameters, given by the number of vertex-deletions needed for a
graph to admit a certain minor-closed property belong all to this class.  Since no known
nonuniformly fixed-parameter tractable problem is W[1]-hard, scientists conjecture that FPT is equal
to nonuniformly FPT.

\subsection{Preliminaries}

\section{Formal results and theories}
\subsection{Graph minors and Robertson and Seymour's theorem}
\subsection{Non-uniformly fixed parameter tractability}
\subsection{Vertex deletion distance}

\section{Problems from the book}
Note that we gave a proof sketch of .. in the proof of ..

\bibliographystyle{plain}
\bibliography{ref.bib}
\end{document}
