% Vorlage f�r Seminar-Ausarbeitungen
%
% Dateiname: Vorlage.tex
% zuletzt ge�ndert: 11. Februar 2016
% Autorin: Nicole Schweikardt

% Definition der Dokument-Klasse, in diesem Falle die Klasse SeminarAusarbeitung,
% die in der Datei SeminarAusarbeitung.cls bereitgestellt wird
\documentclass{SeminarAusarbeitung}

\title{Graph minors\\The Robertson-Seymour theorem}
\author{Narek Bojikian}
\seminar{Parameterized Complexity}
\semester{Winter semester 20/21}
\leitung{Prof.\ Dr.\ Stefan Kratsch} 
\institut{Institut f�r Informatik}
\universitaet{Humboldt-Universit�t zu Berlin}

\usepackage{bbold}
\usepackage{subcaption}
\usepackage{xcolor}
\usepackage{graphicx}
\usepackage{float}
\usepackage{natbib}
\usepackage{caption}
\usepackage{tikz}
\usepackage{bussproofs}
\usepackage{proof}
\usepackage{cleveref}
\usepackage[shortlabels]{enumitem}
\usepackage{xcolor}

\newcommand {\mfirst}{minor\textsubscript{1} }
\newcommand {\msecond}{minor\textsubscript{2} }

\newcommand {\forb}{\operatorname{Forb}}
\newcommand {\splanar}{\mathcal{S}_{\operatorname{PLANAR}}}
\newcommand {\sforest}{\mathcal{S}_{\operatorname{FOREST}}}

\newcommand {\red}{\color{red}}

\begin{document}
\maketitle
\section{Introduction}
Robertson and Seymour's theorem is a very celebrated result in theoretical computer science, namely the
theorem of Robertson and Seymour. This theorem finds applications in different areas of theoretical
computer science and combinatorics ranging from extremal graph theory to computational geometry, approximation
algorithms and fixed-parameter analyses. This theorem was the result of a series of twenty papers
written by Robertson and Seymour and positively answers the conjecture of Wagner
\cite{robertson2004graph}. In this report we summarize the applications of this theorem to
parameterized complexity reintroducing the notion of nonuniformly fixed-parameter tractability.

The theorem implies that graph properties that are
preserved by taking minors can be described by a finite set of (minimal) forbidden minors. However,
the proof of the theorem is non-constructive and does not deliver a way to find such a list. Such
lists are only known for a few minor-closed classes of graphs such as planar graphs (Kuratowski's
Theorem), outer-planar graphs and forests. For some graph-classes such lists are only partially known
and can be arbitrary large for example, Toroidal graphs (graphs that can be embedded on a torus)
have at least 250,815 known forbidden minors. On the other hand, Robertson and Seymour provide an
algorithm for testing whether a given graph is a minor of another. Given the complete list of
forbidden minors of some class $\mathcal{C}$, this results yield a polynomial time algorithm to test
whether a given graph belongs to this class or not. 

One application of graph minor theorem, is in the field of parameterized complexity. As discussed
above, some properties of graphs stay preserved in all minors of a graph. We have seen that
polynomial time algorithms for testing such properties exist. Under these properties we consider
some graph parameters that have their values preserved in all minors of a given graph. Hence, there
exists an algorithm that given a graph $G$ and an integer value $k$, it tests whether the value of
this parameter in $G$ is at most $k$ in polynomial time.  However, since the list of forbidden
minors must be hard-coded into the algorithm and we do not know whether an algorithm exists that
generates such a list for each value of the parameter, deciding whether $G$ has the parameter value
$k$ is not quite fixed-parameter tractable. Still, knowing that such an algorithm exists is a strong
notion that a given problem might be fixed-parameter tractable. We will call problems that admit
such algorithms nonuniformly fixed-parameter tractable. As we shall see later in this report, a
family of parameters, given by the number of vertex-deletions needed for a graph to admit a certain
minor-closed property belong all to this class. Since no known nonuniformly fixed-parameter
tractable problem is W[1]-hard, scientists conjecture that FPT is equal to nonuniformly FPT.

\subsection{Preliminaries}
\begin{definition}
	Let $G = (V, E)$ be an undirected graph and $e = \{u, v\}$ an edge in $G$. We say that $G' =
	(V', E')$ is the graph resulting from $G$ by contracting $e$, where $G'$ results from $G$ by
	removing $u, v$ and adding a new vertex $w$ adjacent to all neighbors of $u$ and to all
	neighbors of $v$. Mathematically speaking $V' = V \setminus \{u, v\} \cup \{w\}$ for a new
	vertex $w$ and $E' = E \setminus \{\{x, y\} \colon x \in \{u, v\}\} \cup \{\{w, y\} \colon
	\{u, y\} \in E \text{ or } \{v, y\} \in E\}$.
\end{definition}
\begin{definition}\label{def:minor1}
	We say that a graph $H$ is a minor of a graph $G$ ($H \leq_m G$) if and only if $H$ results
	from $G$ by vertex and edge deletions and edge contractions. In other words, $H$ is a minor
	of $G$ if and only if there exists a homeomorphism between $H$ and a subgraph of $G$.
\end{definition}
Equivalently, Minors can be given by the following definition. We introduce it and briefly show the
equivalence between both definitions.
\begin{definition}\label{def:minor2}
	A graph $H$ is a minor of another graph $G$, if for each $h \in V(H)$ there exists a branch
	set $V_h \subseteq V(G)$, such that
	\begin{enumerate}[(a)]
		\item $G[V_h]$ is connected;
		\item for different $g,h \in V(H)$, the branch sets $V_g$ and $V_h$ are disjoint;
		\item and for every $\{gh\} \in E(H)$ there exists an edge $\{v_gv_h\} \in E(G)$
			such that $v_g \in V_g$ and $v_h \in V_h$.
	\end{enumerate}
\end{definition}
Now we show the equivalence of both definitions answering with which \textbf{Exercise 6.12.}.
\begin{proposition}
	Both \cref{def:minor1} and \cref{def:minor2} of a minor are equivalent. That means, for
	\mfirst the construction defined in the first definition and \msecond the one in the second
	definition, a graph $H$ is a \mfirst of a graph $G$ if and only if $H$ is a \msecond of
	$G$.
\end{proposition}
\begin{proof}
	Let $H$ and $G$ be two given graphs. If $H$ is a \msecond of $G$, then $H$ results from $G$ by
	a sequence of deletions and contractions as follows: First remove all vertices that do not
	belong to any $v_h$ for any $h \in V(H)$. Since $V_h$ induce a connected subgraph in $G$, we
	can contract all edges of such a connected-subgraph $G[V_h]$ into a single vertex $h'$. We
	claim that there exists a homomorphism from $H$ to the resulting graph $G'$, where we assign
	each vertex $h$ to $h'$ the vertex resulting from contracting all edges in $G[V_h]$. This is
	true due to the third condition of \cref{def:minor2}. We can turn this homomorphism into an
	isomorphism by removing all extra edges in $G'$ and hence $H$ is a \mfirst of $G$.

	Now assume that $H$ is a \mfirst of $G$ and we have to prove that $H$ is a \msecond of $G$.
	We prove this claim by induction over the sequence of deletions and contractions. Note that
	$G$ is a \msecond of it self, where we set $V_h = \{h\}$ for each  vertex $h \in V(G)$. Now
	assume that the graph $H'$ resulting after $n$ operations is a \msecond of $G$. Let $H$ be the
	graph after the $n+1$st operation. If the $n+1$st operation is a vertex-deletion or an
	edge-deletion, then $H$ is a \msecond of $G$ using the same sets $V_u\colon u \in V(H)$. If it
	is an edge-contraction $\{u, v\}$, let $w$ be the resulting vertex. Then $H$ is a \msecond of
	$G$ using the same sets $V_u \colon u \in V(H) \setminus \{w\}$ and $V_w := V_u \cup V_v$.
\end{proof}

\begin{definition}[Minor-closeness]
	Let $\mathcal{C}$ be a given class of graphs. We call $\mathcal{C}$ minor-closed, if for
	each graph $G \in \mathcal{C}$ it holds that $H \in \mathcal{C}$ for all $H \leq_m G$.
\end{definition}

For example, the class of all forests is minor closed. The class of all planar
graphs and the class of graphs with vertex cover number at most $k$ for some constant value $k$ are
also minor closed. As a reminder, a planar graphs is a graph that can be embedded to the plain, and
an embedding is a crossing-free drawing. The vertex cover number of a graph $G$ is the smallest size
of a set of vertices $S \subseteq V(G)$ such that $S$ is incident to all edges in $G$. In other
words, $G \setminus S$ is an independent set.
%\begin{lemma}
%	The class of all planar-graphs is minor closed.
%\end{lemma}
%\begin{proof}
%	We use \cref{def:minor1} and show the statement inductively over the sequence of operations.
%	For the base case we have no operations. We are done since $G$ is planar by assumption. We
%	assume that $G$ stays planar after $n$ operations. Let $\mathcal{G}$ be an embedding of $G$.
%	If the $n+1$st operation is a vertex- or an edge-deletion, we can remove this vertex or edge
%	from the drawing $\mathcal{G}$. Since deletion cannot introduce intersection, the embedding
%	is still valid. If the $n+1$st operation is a contraction of some edge $\{u, v\}$ and let
%	$w$ be the new vertex. We assume that the given embedding is a straight-line embedding. We
%	can assume that since each planar graph admits a straight-line embedding. We sketch an
%	embedding of the resulting graph. We draw $w$ in the middle of the line-segment that
%	corresponds to $\{u, v\}$. Now let $x_1, \cdots x_k$ be the neighbors of $x$ sorted
%	according to their polar angle from $w$. We draw $k$ line-segments from $w$ in the direction
%	of $x$ close enough to each other and differ by a very small degree. Then we start from the
%	top one and connected to $x_1$. We extend the second line-segment slightly to make and
%	connect it to $x_2$ and continue in the same manner until we connect $w$ to all the
%	neighbors of $u$. We repeat the same procedure again for $v$ ignoring the common neighbors
%	of $u$ and $v$. Note that by very small we mean small enough, that there endpoints can
%	``see'' the exact same points and segments of the original drawing.  Note also that the
%	resulting drawing is not necessarily a straight-line drawing. It is nevertheless a
%	crossing-free drawing and suffices to our needs. A graphical explanation is depicted in
%	\cref{fig:planarcontraction}
%\end{proof}
%\begin{figure}[htpb]
%	\centering
%	\begin{subfigure}{.3\textwidth}
%	\centering
%	\includegraphics[width=\linewidth]{planar-contraction-1.eps}
%	\end{subfigure}
%	\hspace{1cm}
%	\begin{subfigure}{.3\textwidth}
%	\centering
%	\includegraphics[width=\linewidth]{planar-contraction-2.eps}
%	\end{subfigure}
%	\caption{Graphical review of edge-contraction in a planar graph that preserves planarity.}%
%	\label{fig:planarcontraction}
%\end{figure}
%\begin{lemma}
%	Let $k \in \mathbb{N}$ be a given integer. The class of all graphs of vertex-cover number at
%	most $k$ is minor-closed.
%\end{lemma}

\section{Formal results and theories}
In this section we list the main theorems of this report. We start with the celebrated theorem of
Robertson and Seymour and show how to apply this theorem in parameterized analyses. We finish the
section by a general schema that spans a list of different parameters.
\begin{theorem}[Roberson and Seymour \cite{robertson2004graph}]
	The class of all graphs is well-quasi-ordered by the minor relation. That is, in any
	infinite family of graphs, there are two graphs such that one is a minor of the other.
\end{theorem}
Now we state a very crucial corollary of this theorem, that we base the rest of this report upon. We
skip the proof and refer to \cite{DBLP:books/sp/CyganFKLMPPS15} for more details.
\begin{corollary}\label{cor:forb}
	For every minor-closed graph class $\mathcal{G}$ there exists a finite set
	$\forb(\mathcal{G})$ of graphs with the following property: for every graph $G$, grph $G$
	belongs to $\mathcal{G}$ if and only if there does not exist a minor of $G$ that is
	isomorphic to a member of $\forb(\mathcal{G})$.$\hfill\square$
\end{corollary}
For example, for $\splanar$ the family of all planar graph, $\forb(\splanar) := \{K_5, K_{3,3}\}$
and for $\sforest$ the family of all forests we have $\forb(\sforest) := \{C_3\}$ where $K_i$ is a
clique of $i$ vertices and $C_i$ is a simple cycle of $i$ vertices for $i \in \mathbb{N}$.

The following result is also by Roberston and Seymour and is very important to make use of the
previous results algorithmically.
\begin{theorem}[Roberson and Seymour]\label{theo:minortest}
	There exists a computable function $f$ and an algorithm that, for given graphs $H$ and $G$,
	checks in time $f(H)|V(G)|^3$ whether $H \leq_m G$.
\end{theorem}
\cref{cor:forb} and \cref{theo:minortest} provide us with a strong tool to test whether a graph
belongs to a minor-closed class $\mathcal{C}$ in polynomial running time given the set
$\forb(\mathcal{C})$.
\begin{definition}[minor-monotnousity]
	Given a graph parameter $\rho$, we call $\rho$ minor-monotone if for all graphs $G$ and all
	minors $H$ of $G$ ($H \leq_m G$) we have $\rho(H) \leq \rho(G)$.
\end{definition}
Given a minor-monotone parameter $\rho$ and a value $k$, the class $\mathcal{C}_{\rho}^k$ of all
graphs $G$ such that $\rho(G) \leq k$ is minor closed. Hence, there exists a polynomial time
algorithm that tests for a given graph $G$ whether $\rho(G) \leq k$. Note that
$\forb(\mathcal{C}_{\rho}^k)$ must be hard-coded into the algorithm, and hence we do note have an
FPT algorithm that computes $\rho(G)$. However, in this case we call the problem of computing
$\rho(G)$ nonuniformly FPT.

\begin{definition}
	We say that a parameterized problem $Q$ is nonuniformly fixed-parameter tractable if there
	exists a constant $\alpha$, a function $f:\mathbb{N} \rightarrow \mathbb{N}$, and a
	collection of algorithms $\left(\mathcal{A}_k\right)_{k\in\mathbb{N}}$ such that the
	following holds. For every $k \in \mathbb{N}$ and every input $x$, the algorithm
	$\mathcal{A}_k$ accepts input $x$ if and only if $(x, k)$ is a yes-instance of $Q$, and the
	running time of $\mathcal{A}_k$ on $x$ is at most $f(k)|x|^{\alpha}$.
\end{definition}
A problem being nonuniformly FPT is a strong notion that this problem could be FPT, since no known
problem so far is nonuniformly FPT and W[1]-hard at the same time.

\begin{definition}
	Given a graph $G$ and a class $\mathcal{C}$. We say that $G$ has vertex-deletion distance at
	most $d$ from $\mathcal{C}$ for some integer $d\in \mathbb{N}$, if there exists a vertex-set
	$S \subseteq V(G)$ of size at most $d$ such that $G\setminus S \in \mathcal{S}$.

	We define the $\mathcal{G}$ vertex-deletion problem for a given class of graphs
	$\mathcal{G}$ as the problem of finding the smallest integer $k$ such that a given graph $G$
	has vertex-deletion distance at most $k$ from $\mathcal{G}$.
\end{definition}
\begin{lemma}
	The class of all graphs of vertex-deletion distance at most $d$ from a minor-closed class of
	graphs is also minor-closed.
\end{lemma}
\begin{proof}
	Let $G$ be a given graph with vertex-deletion distance at most $d$ from a minor-closed class
	$\mathcal{C}$. Let $H$ be a minor of $G$. We show that $H$ has vertex-deletion distance at
	most $d$ from $\mathcal{C}$ as well. We prove this claim by induction over the length $\ell$
	of the sequence of operations to get $H$ from $G$ according to \cref{def:minor1}. The base
	case where $\ell = 0$ is trivial. Assume that the minor $H'$ resulting after $n$ operations
	admits the given property. Let $H$ be the minor resulting after $n+1$ operations.  Assume
	that $S'$ is a minimum modulator of $H'$. If the $n+1$st operation is a vertex-deletion, if
	the deleted vertex lays in $S'$, then $H$ has a strictly smaller vertex-deletion distance to
	$\mathcal{C}$ than $H'$. Otherwise, $H\setminus S'$ is a minor of $H'\setminus S'$.
	Similarly, if we remove an edge, then $H\setminus S' \leq_m H' \setminus S'$. Now assume
	that the operation is a contraction of an edge $\{u, v\}$ creating a new vertex $w$ in $H$.
	If it has an endpoint in $S'$, we set $S := (S' \setminus \{u, v\})\cup \{w\}$. Otherwise
	set $S := S'$ clearly $H \setminus S \leq_m H' \setminus S'$. Since $\mathcal{C}$ is
	minor-closed, $H$ has distance at most $d$ from $\mathcal{C}$.
\end{proof}
Using the previous lemma, it is straightforward to show that vertex-cover number and feedback vertex
number are minor closed, since they are the vertex-deletion distance to independent-sets and forests
respectively and both graphs are minor-closed. Note that feedback vertex number is the smallest size
of a set of vertices that hits all cycles in a given graph. It follows directly, that feedback
vertex number if nonuniformly FPT.

\begin{corollary}
	The $\mathcal{G}$ vertex-deletion problem for a minor-closed class $\mathcal{G}$ is
	nonuniformly FPT.
\end{corollary}


\section{Problems from the book}
We have already solved 6.12 and 6.14. in the course of the report.\\[2ex]%
%
\noindent \textbf{Exercise 6.16}
We claim that the class $\mathcal{C}_k$ of graphs having a planar supergraph of diameter at most $k$
is minor-closed. Using this claim, \cref{cor:forb} and \cref{theo:minortest} we know that for each
integer $k$ there exists a finite set $\forb(\mathcal{C}_k)$ and an algorithm that tests whether any
graph in this set is a minor of the input graph. If non is, the given instance must be a
yes-instance.

Let $G$ be an arbitrary graph, such that there is a planar supergraph of $G$ of diameter at most
$k$. Let $H$ be a minor of $G$.  We prove the claim by induction over the length $\ell$ of the
sequence of operations from \cref{def:minor1} to get $H$ from $G$. The base case where $\ell = 0$ is
trivial. Now assume that the claim is true for the first $n$ operations and let $H$ be the graph
resulting after the $n$ operations. Assume that $H' \in \mathcal{C}_k$. Let $H$ be the graph that
results after $n+1$ operations. If $H$ results from $H'$ by removing a vertex or an edge, then $H'$
is a supergraph of $H$ and hence, each supergraph of $H'$ is a supergraph of $H$. Assume that $H$
results from $H'$ by contracting some edge $\{u, v\}$ into a new vertex $w$. Let $F'$ be a
supergraph of $H'$ of radius at most $k$ and let $F$ be the graph resulting from $F'$ by contracting
$\{u, v\}$ and call the new vertex $w_F$. We claim that $F$ is isomorphic to a supergraph of $H$ and
has radius at most $k$ which completes the proof.

For the first claim note that contraction is a local operation, and it suffices to prove that for
each edge $\{w, x\}$ in $G$ there is an edge $\{w_F, x\}$ in $F$. But such an edges implies that at
lest one of $\{u, x\}$ or $\{v, x\}$ is present in $H'$ and hence, in $F'$. Hence, by the definition
of contraction, the edge $\{w_F, x\}$ exists in $F$.\\[2ex]%
%
\noindent \textbf{Exercise 6.17}
Similar to Exercise 6.16, it suffices to prove, that the class $\mathcal{C}$ of graphs that admit an
embedding to $R^3$ such that any pairwise linked family of cycles in $G$ has size at most $k$ is
minor closed.  The rest of the proof follow exactly the same.

Similarly to Exercise 6.16, we prove the claim by induction over the length $\ell$ of the sequence
of operations from \cref{def:minor1} to derive a minor $H$ from a graph $G$. The base case where
$\ell=0$ is trivial. Assume the correctness of the statement for $\ell = n$ and let $H$ be the graph
resulting from $G$ after $n + 1$ operations. Let $H'$ be the graph after $n$ operations. By
induction hypothesis, $H' \in \mathcal{C}$. We need to prove that $H \in \mathcal{C}$. If $H$
results from $H'$ be removing a vertex or an edge, we can use the same embedding we used for $H'$
and remove the corresponding vertex or edge from it. Assume that the resulting embedding contains a
pairwise linked family of cycles of size strictly greater than $k$, then this family must have
existed in $H'$ since we only add objects to the drawing. Which is a contradiction to the assumption
that $H' \in \mathcal{C}$.

Now assume that $H$ results from $H'$ by an edge-contraction. We can turn an embedding of $H'$ to an
embedding of $H$ by drawing the resulting vertex on an arbitrary point on one of the endpoints of
the contracted edge, connecting all neighbors of this endpoint to the new vertex using the exact
same curves and the neighbors of the other endpoints using the same curves where we remove a very
small part of each of these curves and continue each such curve along and very close to the
curve corresponding to the contracted edge to reach the new vertex. Assume that there is a pairwise
linked family of cycles in $H$ of size strictly more than $k$. This family must contain the vertex
resulting from the contraction,
else it would have existed in $H'$. But then we can replace this new vertex either by the contracted
curve or by one of its endpoints getting a similar family in the given drawing oh $H'$ which
contradicts the assumption, that the given drawing is linkless.\\[2ex]%
%
\noindent \textbf{Exercise 6.19} Let $\mathcal{C}_k$ be the set of graphs containing $k$
vertex-disjoint cycles and let $G_k$ be a disjoint union of $k$ simple cycles of 3 vertex $C_3$.
Given a graph $G$. We claim that $G \in \mathcal{C}_k$ if and only if $G_3 \leq_m G$. Using this
claim and \cref{theo:minortest}, we known that for each $k$ there is an algorithm that given a graph
$G$ tests in polynomial time in $G$ whether $G_k$ is a minor of $G$ or not.

\noindent $\implies:$ Given such a graph $G$, let $C^1, \dots C^k$ be a set of $k$ vertex-disjoint
cycles in $G$. We contract all be three edges of each such cycles, remove all vertices not appearing
on any of these cycles and edges that connect cycles getting a disjoint union of k $C_3$ cycles as a
minor of $G$.

\noindent $\impliedby:$ Again we prove this claim by induction over the length $\ell$ of the
sequence of operations to get the minor from $G$. The base case $\ell = 0$ and the given graph is
already a disjoint union of $k$ copies of $C_3$ and hence is a yes-instance. Now assume $G :=
H_{n+1}, H_{n}, \cdots {H_0} := G_k$ is a sequence of graphs resulting from $n+1$ operations from
\cref{def:minor1}. $H_n$ has a vertex-disjoint set of $k$ cycles by induction hypothesis. If $H_n$
results from $G$ by deleting a vertex of an edge, then the same set of cycle exists in $G$.
Otherwise it results by edge contractions. The emerging vertex appears on at most one cycle, we can
turn this cycle into a cycle in $G$ by replacing this vertex either by the contracted edge or by one
of its endpoints getting a vertex-disjoint $k$ cycles in $G$.


\bibliographystyle{plain}
\bibliography{ref.bib}
\end{document}
